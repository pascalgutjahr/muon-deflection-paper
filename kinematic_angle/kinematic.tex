\documentclass[a4paper]{article}

\usepackage[utf8]{inputenc}
\usepackage[T1]{fontenc}
\usepackage[english]{babel}

\usepackage[intlimits]{amsmath}
\usepackage{cmupint,amssymb,booktabs}
\usepackage[separate-uncertainty=true]{siunitx}

\usepackage{natbib}

\begin{document}
\bibliographystyle{plainnat}
\title{Calculation of the kinematic angle in charged current $\nu_\mu$
  interactions}
\author{Alexander Sandrock}
\date{August 11, 2022}
\maketitle

\section{Charged current $\nu N$ cross section}
The cross section for charged current neutrino-nucleon interaction is given by 
\citep{block2013b}
\begin{multline}
  \frac{d^2\sigma_\text{CC}^{\nu (\overline \nu)}}{dx\ dQ^2} (E_\nu, Q^2, x)
    = \frac{G_F^2}{4 \pi} \left(\frac{M_W^2}{Q^2 + M_W^2}\right)^2 \\
  \times \frac{1}{x} \left[ F_2^{\nu (\overline \nu)}
    \pm x F_3^{\nu (\overline \nu)} + (F_2^{\nu (\overline \nu)} \mp
    F_3^{\nu (\overline \nu)}) \left(1 - \frac{Q^2}{2 m x E_\nu}\right)^2
    - \left(\frac{Q^2}{2 m x E_\nu}\right)^2 \right];
\end{multline}
at high energies $E_\nu$, the dominant contributions to the cross section come
from small values of $x$, where $F_3$ is subdominant to $F_2$. In addition, to
leading order in the strong coupling, the longitudinal structure function $F_L$
vanishes. The structure function $F_2^{\nu (\overline{\nu})}$ is given in terms
of the structure function $F_2^{\gamma p}$ known from deep inelastic scattering
experiments as \citep{block2013a,block2013b}
\begin{equation}
  F_2^{\nu (\overline \nu)} = \frac{45}{11} F_2^{\gamma p}
\end{equation}
plus small non-singlet combinations; here, only the result for $n_f = 5$ active
flavors is given, as small values of $Q^2$ are suppressed by the mass of the $W$
boson and $m_b \ll M_W$.

For the purposes of this document, we use the Froissart-bounded fit of the
structure function $F_2^{\gamma p}$ from \citet{block2014}
\begin{equation}
  F_2^{\gamma p} (x, Q^2) = D(Q^2) (1 - x)^n \left[ C(Q^2)
    + A(Q^2) \ln \frac{Q^2/x}{Q^2 + \mu^2}
    + B(Q^2) \ln^2 \frac{Q^2/x}{Q^2 + \mu^2} \right],
\end{equation}
where
\begin{align}
  A(Q^2) &= a_0 + a_1 \ln \left(1 + \frac{Q^2}{\mu^2} \right)
    + a_2 \ln^2 \left(1 + \frac{Q^2}{\mu^2} \right), \\
  B(Q^2) &= b_0 + b_1 \ln \left(1 + \frac{Q^2}{\mu^2} \right)
    + b_2 \ln^2 \left(1 + \frac{Q^2}{\mu^2} \right), \\
  C(Q^2) &= c_0 + c_1 \ln \left(1 + \frac{Q^2}{\mu^2} \right), \\
  D(Q^2) &= \frac{Q^2 (Q^2 + \lambda M^2)}{(Q^2 + M^2)^2},
\end{align}
and the values of the fit parameters are given in Table~\ref{table:fit}.
\begin{table}
  \caption{Fit values for the parameter of the structure function
    $F_2^{\gamma p}$ from \citet{block2014}.}
  \begin{center}
    \begin{tabular}{lc}
      \toprule
      Parameters & Values \\
      \midrule
      $M^2$ & \SI{0.753(068)}{GeV^2} \\
      $\mu^2$ & \SI{2.82(029)}{GeV^2} \\
      $a_0$ & \num{8.205(4620)e-4} \\
      $a_1$ & \num{-5.148(0819)e-2} \\
      $a_2$ & \num{-4.725(1010)e-3} \\
      $b_0$ & \num{2.217(0142)e-3} \\
      $b_1$ & \num{1.244(0086)e-2} \\
      $b_2$ & \num{5.958(2320)e-4} \\
      $c_0$ & \num{0.255(0016)} \\
      $c_1$ & \num{1.475(303)e-1} \\
      $n$ & \num{11.49(099)} \\
      $\lambda$ & \num{2.430(0153)} \\
     \bottomrule
    \end{tabular}
  \end{center}
  \label{table:fit}
\end{table}

\section{Kinematic relations of Bjorken $x$, inelasticity $y$ and momentum
  transfer $Q^2$ with the angle $\theta$ between neutrino and charged lepton}
The Bjorken scaling variable can be expressed as
\begin{equation}
  x = \frac{Q^2}{2 m y E_\nu},
\end{equation}
where $m$ is the nucleon mass and $y = (E - E')/E$ is the inelasticity, i.\,e.
the relative energy transfer from the neutrino to the nucleon. The change of
variables from $x, Q^2$ to $y, Q^2$ is therefore performed as
\begin{equation}
  \frac{d^2 \sigma_\text{CC}^{\nu (\overline \nu)}}{dy\ dQ^2} =
  \frac{x}{y} \frac{d^2 \sigma_\text{CC}^{\nu (\overline \nu)}}{dx\ dQ^2}.
\end{equation}
The angle $\theta$ enters into the momentum transfer
 $Q^2 = -(k - k')^2$, where $k, k'$ denote the 4-momenta of the neutrino and
charged lepton, respectively, such that
\begin{equation}
  \cos \theta (E_\nu, y, Q^2) = 1 - \frac{Q^2}{2 E_\nu^2 (1 - y)}
\end{equation}
for lepton masses $m_l^2 \ll Q^2$.

To determine the average scattering angle $\theta$ between neutrino and charged
lepton as a function of neutrino energy, we have to calculate
\begin{equation}
  \langle \cos\theta\rangle = \frac{\int dy\ dQ^2\ \cos \theta
    [d^2\sigma/(dy\ dQ^2)]}{\int dy\ dQ^2\ [d^2\sigma/(dy\ dQ^2)]};
\end{equation}
by analogously calculating $\langle \cos^2 \theta\rangle$, the scatter of the
scattering angle can be determined as $\delta \cos \theta = \sqrt{\langle \cos^2
\theta\rangle - \langle \cos \theta\rangle^2}$.

To determine the average scattering angle as a function of the final muon
energy, one has to introduce a prior on the neutrino energy and express the
inelasticity via $y = (E_\nu - E_\mu)/E_\nu$; an obvious choice for the prior
would be a power law $N(E_\nu) \propto E^{-\gamma}$ with spectral index $2 \geq
\gamma < 4$.

\section{Numerical results}

\bibliography{kinematic}
\end{document}
