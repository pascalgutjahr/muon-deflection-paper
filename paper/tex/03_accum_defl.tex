\section{Accumulated Muon Deflection}\label{sec:accum_defl}

As shown in Section~\ref{sec:defl_per_int}, the deflection per interaction 
is lower $\SI{1}{\degree}$. Since these deflections accumulate along the 
propagation distance, the angle between the incoming muon and the outgoing 
muon direction is simulated to study a limit on the angular reconstruction.
At first, the deflections in PROPOSAL are compared to 
the tools MUSIC \cite{MUSIC,comparison_MUSIC_GEANT4_2009} and GEANT4 \cite{GEANT4}.
MUSIC (MUon SImulation Code) is a tool to simulate the propagation of muons 
through media like rock and water considering the same energy losses as in 
PROPOSAL. Also, the losses are divided into continuous and stochastic 
energy losses by a relative energy cut. 
\begin{itemize}
    \item this is MUSIC: maybe use a table for the cross-section comparison?
    \item electron pair production:  Kelner-..[\cite{epair_kelner,epair_kokoulin_petrukhin}], deflection [\cite{Van_Ginneken}]
    \item bremsstrahlung: Kelner, Kokoulin andPetrukhin
    [\cite{Bremsstrahlung_KKP}], deflection [\cite{Van_Ginneken}] 
    \item inelastic scattering: shlepin and Bugaev [\cite{nulcint_bugaev_Shlepin}], \cite{bugaev_1980_defl,bugaev_1981_defl} (überprüfe, ob genau diese referenz auch im GEANT4 verwendet wird!!!)
    \item ionization: treated as stochastic process (Bethe-Bloch formula) including knock-on electron production, but no deflection
    \item scattering: gaussian approx \cite{HIGHLAND_1975}
\end{itemize}

\textcolor{red}{add short description and settings of MUSIC}
\textcolor{red}{add short description and settings of GEANT4}
\textcolor{red}{was machen MUSIC und GEANT4 mit ioniz?}
\textcolor{red}{Moliere beschreibt nur die Winkeländerung, jedoch nicht die laterale Verschiebung. Wie machen wir das mit der 
laterale Verschiebung in PROPOSAL für Moliere? Gaussian approx berücsichtigt die Verschiebung}

\begin{itemize}
    \item this is GEANT4
    \item electron pair production: 
    \item bremsstrahlung:
    \item inelastic scattering: cross-section \cite{Borog:1975_inelastic}, deflection \cite{Borog:1977_inelastic,Borog:1975_inelastic}
    \item ionization:
    \item scattering:
\end{itemize}

\begin{itemize}
    \item this is PROPOSAL
    \item electron pair production: KelnerKokoulinPetrukhin (Proc. 12th ICCR (1971), 2436) with corrections for the interaction with atomic electrons (Phys. Atom. Nucl. 61 (1998), 448) [\cite{}], deflection \cite{Van_Ginneken}
    \item bremsstrahlung: KelnerKokoulinPetrukhin (Preprint MEPhI (1995) no. 024-95) and (Phys. Atom. Nucl. 62 (1999), 272) [\cite{}], deflection \cite{Van_Ginneken,GEANT4}
    \item inelastic scattering: AbramowiczLevinLevyMaor97 (arXiv::hep-ph/9712415) \cite{} with shadowing ButkevichMikheyev (JETP 95 (2002), 11) \cite{}, deflection \cite{Van_Ginneken, Borog:1977_inelastic,Borog:1975_inelastic}
    \item ionization: BetheBlochRossi Ionization described by Bethe-Bloch formula with corrections for muons and taus by (B. B. Rossi. Prentice-Hall, Inc., Englewood Cliffs, NJ, 1952) [\cite{}], deflection [direct calculation using four-momentum transfers]
    \item scattering: \cite{HIGHLAND_1975,moliere_scattering}
\end{itemize}


\begin{figure}
    \centering
    \subcaptionbox{
        The accumulated deflection $\theta_{\mathrm{acc}}$ in degree is very similar in all cases.
        \label{fig:compare_MUSIC_degree}}
        {\includegraphics[width=0.48\textwidth]{figures/compare_MUSIC_angle_paper.pdf}}
    \subcaptionbox{
        The lateral displacement $x$ in meter results in two cases, which depend 
        on the scattering method. Moliére scattering leads to larger distances.
        \label{fig:compare_MUSIC_dist}}
        {\includegraphics[width=0.48\textwidth]{figures/compare_MUSIC_dist_paper.pdf}}
    \caption{A comparison of the results of MUSIC, GEANT4 and PROPOSAL is presented for $\num{1000000}$ negative charged muons propagated with 
    $E_{\text{i}} = \SI{2}{\tera\electronvolt}$ over a distance of 
    $\SI{3}{\kilo\meter}$ in water. A $\texttt{v\_cut} = 0.001$ is set. In PROPOSAL, 
    bremsstrahlung and photonuclear interaction are parametrized by 
    Van Ginneken (vG) and GEANT4. Detailed information are given in 
    Table~\ref{tab:compare_MUSIC}. The results for MUSIC and GEANT4 are taken from 
    \cite{comparison_MUSIC_GEANT4_2009}.}
    \label{fig:compare_MUSIC}
\end{figure}



\begin{table}
    \small
    \centering
    \caption{The survival probability $p_{\text{s}}$, the mean survived muon 
    energy $\overline{E}_{\text{f}}$, the mean scattered angle $\overline{\theta}$ 
    and the mean displacement $\overline{x}$ are presented for all cases from 
    Figure~\ref{fig:compare_MUSIC}. For all means, the standard deviation is given.
    The largest deflection and displacement result in the tool GEANT4, which has the lowest mean survived energy. The lower the energy, the larger the deflection.}
    \begin{tabular}{l|cc|cccc}
        \toprule
        & & & \multicolumn{4}{c}{PROPOSAL} \\
        &  & & \multicolumn{2}{c}{Molière} & \multicolumn{2}{c}{Highland} \\
        & MUSIC & GEANT4 & vG & GEANT4 & vG & GEANT4 \\
        \midrule
        $p_{\text{s}}\,/\,\si{\percent}$ & 77.9 & 79.3 &  \multicolumn{4}{c}{77.9}\\
        $\overline{E}_{\text{f}}\,/\,\si{\giga\electronvolt}$ & 323 & 317 & \multicolumn{4}{c}{331$\pm$178} \\
        $\overline{\theta}\,/\,\si{\degree}$ & 0.22 & 0.27 & 0.24$\pm$0.45 & 0.24$\pm$0.45 & 0.22$\pm$0.35 & 0.22$\pm$0.35   \\
        $\overline{x}\,/\,\si{\meter}$ & 2.6 & 3.3 & 2.9$\pm$2.6 & 2.9$\pm$2.6 & 2.7$\pm$1.6 & 2.7$\pm$1.7  \\
     \bottomrule
    \end{tabular}
    \label{tab:compare_MUSIC}
\end{table}




For current analyses, it is important to study the impact of the muon 
deflection on the angular resolution to estimate a reconstruction uncertainty.
For this purpose, four different initial energies 
from $E_{\text{i}} = \SI{10}{\tera\electronvolt}$ to 
$E_{\text{i}} = \SI{10}{\peta\electronvolt}$ are used and the final 
energy is set to $E_{\text{f,\,min}} \geq \SI{10}{\giga\electronvolt}$ with 
$E_{\text{f,\,min}} < E_{\text{i}}$ for each simulation. To compare the results of 
a total of $\num{36}$ simulations, the median of the deflection distribution 
with a $\SI{95}{\percent}$ central interval is presented in 
Figure~\ref{fig:fit_median}.
The lower the final muon energy, the larger the accumulated deflection. 
For energies $E_{\text{f}} = \SI{1}{\peta\electronvolt}$, the median deflection 
is lower than $\SI{e-3}{\degree}$. For energies $E_{\text{f}} = \SI{10}{\giga\electronvolt}$, 
angles larger than $\SI{1}{\degree}$ are possible. Deflections for a typical energy range 
for astrophysical neutrinos of $E_{\text{f}} \approx \SI{500}{\giga\electronvolt} - \SI{5}{\tera\electronvolt}$
\cite{jan?} results in small overlap with the angular resolution of KM3NeT 
\cite{KM3NeT_Resolution2016}. The resolution of IceCube is a bit worse and 
therefore not affected \cite{IceCube_Resolution2021}. Since these simulations are done 
in ice, the same simulations are done in water. The deviations of the medians
are less than $\SI{1}{\percent}$ for all energies.

Furthermore, in Figure~\ref{fig:fit_median} there are only $12$ medians visible, 
instead of $36$ which is the total amount of all simulations. This result points 
out that the total deflection of a muon 
primarily depends on the final muon energy. Hence, the reconstructed muon 
energy in a detector can be used to estimate a theoretical deflection. For this 
purpose, the following fit-function 
\begin{equation}
     f(x) = a \cdot x^3 + b \cdot x^2 + c \cdot x + d \,,
    \label{eqn:fit_median}
\end{equation}
can be applied with the parameters 
\begin{align}
    a =& +0.024 \pm 0.001\,,  & c =& +0.379 \pm 0.057\,,\\
    b =& -0.312 \pm 0.016\,,  & d =& -0.216 \pm 0.058\,,
\end{align}
in the logarithmic space via 
\begin{align}
    g(x) =& 10^{f(x)}\,, & x =& \log_{10}\left(\frac{E_{\text{f}}}{\si{\giga\electronvolt}}\right)\,.
\end{align}



\begin{figure}
    \centering 
    \includegraphics[width=0.96\textwidth]{figures/fit_median_defl_cut_10percent_only_poly.pdf}
    \caption{The median of the accumulated deflection $\theta_{\text{acc}}$ in degree 
    with a $\SI{95}{\percent}$ 
    central interval is shown for four different initial energies $E_{\text{i}}$. 
    Each data set includes more than $\num{50000}$ events with the requirement 
    that the true final particle energy $E_{\text{f}}$ is maximum 
    $\SI{10}{\percent}$ below the set final energy $E_{\text{f,\,min}}$,   
    $E_{\text{f}} > E_{\text{f,\,min}} \cdot 0.9$. The energy cuts are $\texttt{e\_cut} = \SI{500}{\mega\electronvolt}$ and $\texttt{v\_cut} = 0.05$ and 
    Moliére scattering is chosen. Simulations are done in ice, the deviation 
    of the medians in a water based simulation are less than $\SI{1}{\percent}$.
    Since the medians overlap for different initial energies, there is no 
    strong impact of the initial energy on the median deflection. These 
    medians can be fit by a third degree polynomial in the log-space as 
    shown in Equation~\ref{eqn:fit_median}. For energies 
    $E_{\text{f}} \approx \SI{500}{\giga\electronvolt} - \SI{1}{\tera\electronvolt}$, there is a minimal influence of deflection on the angular resolution of 
    KM3NeT \cite{KM3NeT_Resolution2016}. The resolution of IceCube is not 
    impacted \cite{IceCube_Resolution2021}.}
    \label{fig:fit_median}
\end{figure}