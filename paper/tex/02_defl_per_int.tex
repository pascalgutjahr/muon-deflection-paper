\section{Muon Deflection per Interaction}\label{sec:defl_per_int}
The tool PROPOSAL propagates charged leptons and photons through media und is 
used in this paper to analyze the deflection of muons. For this purpose, 
negative charged muons $\mu^-$ are propagated through the media ice and water 
to estimate the deflection for IceCube and KM3NeT/ARCA. All muon interaction types as bremsstrahlung, photonuclear interaction, electron pair production, 
ionization and decay are provided. The interaction processes are sampled by their cross section. Since bremsstrahlung interactions can be 
arbitrary small due to a massless exchange particle, the photon, an energy cut is introduced to avoid an infinite number of bremsstrahlung interactions 
and furthermore to speed up the propagation process. 
The limit is applied with 
\begin{equation}
    E_{\text{loss,min}} = \min{(E \cdot \texttt{v\_cut}, \texttt{e\_cut})}\,,
\end{equation}
using two parameters - a relative and total energy cut denoted as 
$\texttt{v\_cut}$ and $\texttt{e\_cut}$. The uncertainties are small 
for a relative energy cut $\texttt{v\_cut}\ll 1$ \cite{}.
A sampled energy loss 
$E_{\text{loss}} < E_{\text{loss,\,min}}$ builds an integrated part of a 
stochastic energy loss referred to as continuous energy loss. The 
propagation process is defined by an initial energy $E_{\text{i}}$ and 
two stopping criteria - a final energy $E_{\text{f,\,min}}$ and a 
maximum propagation distance $d_{\text{min}}$. If the last interaction of 
a propagation is sampled by a stochastic interaction, the true final energy 
$E_{\text{f}}$ and the 
propagation distance $d$ can become lower than the required limits. 
The deflections for stochastic interactions are parametrized by Van Ginneken 
in \cite{Van_Ginneken} with a direct calculation of the deflection in 
ionization using four-momentum conservation. 
Furthermore, there are parametrizations for deflections given in GEANT4 \cite{GEANT4_manual} for bremsstrahlung and photonuclear interaction, which 
are available in PROPOSAL, too.
To estimate the deflection along 
a continuous energy loss, multiple scattering established by Moliére 
\cite{moliere_scattering} and the gaussian approximation by Highland 
can be chosen \cite{highland_scattering}. 
The latest updates with a detailed description of the whole tool can be found 
in \cite{phd_soedingrekso}.
All simulations are done with PROPOSAL $7.3.0$.


The deflections per interaction are presented in Figure~\ref{fig:defl_per_int} 
for each interaction type and the total amount. A single deflection 
extend over several orders of magnitude with a median of $\SI{1.5e-6}{\degree}$
and a $\SI{95}{\percent}$ central interval of $[\SI{1.6e-7}{degree}, \,\SI{2.8e-4}{degree}]$.

\begin{figure}
    \centering 
    \includegraphics[width=0.8\textwidth]{figures/1PeV_1TeV_1000events.pdf}
    \caption{The propagation is done for $\num{1000}$ 
    muons from $E_{\text{i}} = \SI{1}{\peta\electronvolt}$ to $E_{\text{f,\,min}} = \SI{1}{\tera\electronvolt}$ using $\texttt{e\_cut} = \SI{500}{\mega\electronvolt}$ and $\texttt{v\_cut} = 0.05$. For multiple scattering 
    the Moliére parametrization is used. Details are presented in 
    Table~\ref{tab:defl_per_int}.}
    \label{fig:defl_per_int}
\end{figure}

\begin{table}
    \centering 
    \caption{The medians of deflections per interaction from Figure~\ref{fig:defl_per_int} are presented for each interaction type and the total distribution with the upper and lower limits of the $\SI{95}{\percent}$ 
    central content levels.}
    \begin{tabular}{ccccc}
        \toprule 
        brems & nuclint & epair & ioniz & total \\
        $\theta\,/\,\SI{e-5}{\degree}$ & $\theta\,/\,\SI{e-4}{\degree}$ & $\theta\,/\,\SI{e-6}{\degree}$ & $\theta\,/\,\SI{e-5}{\degree}$ & $\theta\,/\,\SI{e-6}{\degree}$\\
        \midrule 
        $3.8_{-0.1}^{+297}$ & $1.2_{-0.4}^{+96}$ & $1.3_{-0.2}^{+42}$ & $4.4_{-0.1}^{+181}$& $1.5_{-0.2}^{+279}$\\ 
        \bottomrule
    \end{tabular}
    \label{tab:defl_per_int}
\end{table}