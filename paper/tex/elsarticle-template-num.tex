%% 
%% Copyright 2007-2020 Elsevier Ltd
%% 
%% This file is part of the 'Elsarticle Bundle'.
%% ---------------------------------------------
%% 
%% It may be distributed under the conditions of the LaTeX Project Public
%% License, either version 1.2 of this license or (at your option) any
%% later version.  The latest version of this license is in
%%    http://www.latex-project.org/lppl.txt
%% and version 1.2 or later is part of all distributions of LaTeX
%% version 1999/12/01 or later.
%% 
%% The list of all files belonging to the 'Elsarticle Bundle' is
%% given in the file `manifest.txt'.
%% 

%% Template article for Elsevier's document class `elsarticle'
%% with numbered style bibliographic references
%% SP 2008/03/01
%%
%% 
%%
%% $Id: elsarticle-template-num.tex 190 2020-11-23 11:12:32Z rishi $
%%
%%
\documentclass[preprint,12pt]{elsarticle}

%% Use the option review to obtain double line spacing
%% \documentclass[authoryear,preprint,review,12pt]{elsarticle}

%% Use the options 1p,twocolumn; 3p; 3p,twocolumn; 5p; or 5p,twocolumn
%% for a journal layout:
%% \documentclass[final,1p,times]{elsarticle}
%% \documentclass[final,1p,times,twocolumn]{elsarticle}
%% \documentclass[final,3p,times]{elsarticle}
%% \documentclass[final,3p,times,twocolumn]{elsarticle}
%% \documentclass[final,5p,times]{elsarticle}
%% \documentclass[final,5p,times,twocolumn]{elsarticle}

%% For including figures, graphicx.sty has been loaded in
%% elsarticle.cls. If you prefer to use the old commands
%% please give \usepackage{epsfig}

%% The amssymb package provides various useful mathematical symbols
\usepackage{amssymb}
%% The amsthm package provides extended theorem environments
\usepackage{amsmath}

%% The lineno packages adds line numbers. Start line numbering with
%% \begin{linenumbers}, end it with \end{linenumbers}. Or switch it on
%% for the whole article with \linenumbers.
\usepackage{lineno}
\linenumbers

% Load custom packages
\usepackage[
    locale=US,
    separate-uncertainty=true,
    per-mode=fraction,    
]{siunitx}
\sisetup{math-micro=\text{µ},text-micro=µ}
\usepackage{xcolor}
\usepackage{subcaption}
\usepackage{hyperref}

\usepackage{float}
\usepackage{scrhack}
\floatplacement{figure}{htbp}
\floatplacement{table}{htbp}

\usepackage{multirow}

% schöne Tabellen
\usepackage{booktabs}
\usepackage{multirow}

% reviewing
\usepackage{todonotes}
\newcommand{\alexander}[1]{\todo[inline,color=green!40]{\textit{Alexander:} #1}}

\journal{Astroparticle Physics}

\begin{document}

\begin{frontmatter}

%% Title, authors and addresses

%% use the tnoteref command within \title for footnotes;
%% use the tnotetext command for theassociated footnote;
%% use the fnref command within \author or \address for footnotes;
%% use the fntext command for theassociated footnote;
%% use the corref command within \author for corresponding author footnotes;
%% use the cortext command for theassociated footnote;
%% use the ead command for the email address,
%% and the form \ead[url] for the home page:
%% \title{Title\tnoteref{label1}}
%% \tnotetext[label1]{}
%% \author{Name\corref{cor1}\fnref{label2}}
%% \ead{email address}
%% \ead[url]{home page}
%% \fntext[label2]{}
%% \cortext[cor1]{}
%% \affiliation{organization={},
%%             addressline={},
%%             city={},
%%             postcode={},
%%             state={},
%%             country={}}
%% \fntext[label3]{}

% \title{Simulate muon deflection with PROPOSAL}
\title{Simulation of theoretical deflection uncertainty on directional 
muon reconstruction using PROPOSAL}

%% use optional labels to link authors explicitly to addresses:
%% \author[label1,label2]{}
%% \affiliation[label1]{organization={},
%%             addressline={},
%%             city={},
%%             postcode={},
%%             state={},
%%             country={}}
%%
%% \affiliation[label2]{organization={},
%%             addressline={},
%%             city={},
%%             postcode={},
%%             state={},
%%             country={}}

\author[TUDortmund]{Pascal Gutjahr}

\affiliation[TUDortmund]{organization={Dept. of Physics, TU Dortmund University},%Department and Organization
            addressline={August-Schmidt-Straße 1}, 
            city={Dortmund},
            postcode={44227}, 
            state={NRW},
            country={Germany}}

\begin{abstract}
For uncertainty estimation of reconstruction of the muon direction, the 
deflection of muons along a propagated path in ice and water is simulated using 
the tool PROPOSAL (Propagator with optimal precision and optimized speed for all leptons), which considers multiple scattering and stochastic deflection. 
The deflection per interaction spans several 
orders of magnitude with $\theta \in [\SI[print-unity-mantissa = false]{e-9}{\degree},\, \SI{1}{\degree}]$ 
for energies $E < \SI{1}{\peta\electronvolt}$ 
and is rather negligible. The accumulated 
deflection of a muon with an energy $E_{\mathrm{\mu}} = \SI{500}{\giga\electronvolt}$ 
after a propagated track is about $\theta_{\text{acc}} = 0.10_{-0.02}^{+0.27}\,\si{\degree}$ 
with a $\SI{95}{\percent}$ central interval, which is in the order of magnitude of 
the directional resolution of 
present neutrino detectors. This deflection can be simply parametrized in 
dependence of the final muon energy. Furthermore, the deflections are compared to the results 
of the propagation tools MUSIC and GEANT4 and all deflections are in good 
agreement. 



\end{abstract}

%%Graphical abstract
\begin{graphicalabstract}
%\includegraphics{grabs}
\end{graphicalabstract}

%%Research highlights
\begin{highlights}
\item Stochastic muon deflection implemented into PROPOSAL
\item Comparison of deflections in MUISIC, GEANT4 and PROPOSAL
\item Third degree polynomial fit of muon deflection 
\item Theoretical estimation of angular uncertainty on directional muon reconstruction 
\end{highlights}

\begin{keyword}
%% keywords here, in the form: keyword \sep keyword

%% PACS codes here, in the form: \PACS code \sep code

%% MSC codes here, in the form: \MSC code \sep code
%% or \MSC[2008] code \sep code (2000 is the default)

Neutrino Astronomy \sep Point-Source-Analysis \sep Angular Resolution 
% (!maximum of 6 keywords!)

\end{keyword}

\end{frontmatter}

%% \linenumbers

%% main text
 
\section{Introduction}\label{sec:introduction}
Neutrinos can cross the universe all the way to earth, since they are 
nearly massless and not charged. On earth, neutrinos can interact 
with a nucleus and produce an electron, muon or tau in the charged current 
via the weak interaction \cite{pdg}. These charged daughter particles are very 
high-energetic and emit Cherenkov light, which can be detected 
by large neutrino telescopes like IceCube \cite{IceCube_Instrumentation} or 
KM3NeT \cite{KM3NeT_Design}. 
By the reconstruction of the direction of the daughter particles, the direction 
of the original neutrino can be inferred. The angular resolution is at its 
best for muons, since they can travel up to a few kilo meters due to their 
large mass, instead of the much lighter electrons and taus - which decay almost 
instantly. Current angular resolutions are 
$\Phi_{\text{I}} > \SI{0.1}{\degree} - \SI{0.3}{\degree}$ for energies 
$E \in [\SI{3}{\tera\electronvolt},\,\SI{3}{\peta\electronvolt}]$ in IceCube 
\cite{IceCube_Resolution2021} 
and 
$\Phi_{\text{K}} < \SI{0.2}{\degree}$ for $E > \SI{10}{\tera\electronvolt}$ in 
KM3NeT/ARCA \cite{KM3NeT_Resolution2021}.
In general, muons do up to several thousand interactions along their propagation, depending 
on their energy, propagation distance and energy cuts, which are mentioned in 
Section~\ref{sec:proposal}.
Since they 
are deflected in each interaction, it is important to study if the accumulated 
deflection along a track impacts the angular resolution of current 
neutrino detectors. 

For this purpose, the lepton propagation 
tool PROPOSAL \cite{koehne2013proposal, dunsch_2018_proposal_improvements} is described in Section~\ref{sec:proposal} first and used to study the muon deflection per interaction in 
Section~\ref{sec:defl_per_int}. Then, the
accumulated deflection is compared to the results of the propagation tools MUSIC and GEANT4 in Section~\ref{sec:accum_defl}. Furthermore, the median deflection is parametrized. The 
paper concludes with a summarized overview in Section~\ref{sec:conclusion}.


% Überlick: 
% Neutrinos kommen aus dem Universum -> fliegen auf Erde -> WW mit Kern 
% -> e,mu,tau entstehen -> geladene hochenergetische Teilchen -> Cherenkov Licht 
% -> kann mit Neutrinodetektoren (wie IceCube und KM3NeT) gemessen werden 
% -> Richtungsrekonstruktion ->
% Rückschluss auf Richtung, aus der das Neutrino entsendet wurde

% -> Myonen haben die beste Richtungsauflösung (Auflösung von IceCube und KM3NeT), 
% da sie aufgrund ihrer hohen Masse 
% eine deutlich größere Strecke zurücklegen können, als elektronen und taus, welche 
% sofort zerfallen -> da myonen entlang ihrer propagierten strecke bis zu 10.000 
% wechselwirkungen durchführen können und sie in jeder einzelnen abgelenkt wird, 
% gilt es nun zu überprüfen, ob die ablenkung der myonen die Richtungsauflösung 
% beeinflusst.

% To determine 
% the position of extraterrestrial neutrino sources more precisely, neutrino 
% experiments are constantly being optimized.

% Neutrinos have a major role to play in the study of the universe. 
% The direction of the original 
% neutrino - emitted by extraterrestrial sources - can be inferred by 
% reconstructing the direction of the daughter 
% particles, which are produced in charged currents.



\section{Muon Deflection per Interaction}\label{sec:defl_per_int}
The tool PROPOSAL propagates charged leptons and photons through media und is 
used in this paper to analyze the deflection of muons. For this purpose, 
negative charged muons $\mu^-$ are propagated through the media ice and water 
to estimate the deflection for IceCube and KM3NeT/ARCA. All muon interaction types as bremsstrahlung, photonuclear interaction, electron pair production, 
ionization and decay are provided. The interaction processes are sampled by their cross section. Since bremsstrahlung interactions can be 
arbitrary small due to a massless exchange particle, the photon, an energy cut is introduced to avoid an infinite number of bremsstrahlung interactions 
and furthermore to speed up the propagation process. 
The limit is applied with 
\begin{equation}
    E_{\text{loss,min}} = \min{(E \cdot \texttt{v\_cut}, \texttt{e\_cut})}\,,
\end{equation}
using two parameters - a relative and total energy cut denoted as 
$\texttt{v\_cut}$ and $\texttt{e\_cut}$. The uncertainties are small 
for a relative energy cut $\texttt{v\_cut}\ll 1$ \cite{}.
A sampled energy loss 
$E_{\text{loss}} < E_{\text{loss,\,min}}$ builds an integrated part of a 
stochastic energy loss referred to as continuous energy loss. The 
propagation process is defined by an initial energy $E_{\text{i}}$ and 
two stopping criteria - a final energy $E_{\text{f,\,min}}$ and a 
maximum propagation distance $d_{\text{min}}$. If the last interaction of 
a propagation is sampled by a stochastic interaction, the true final energy 
$E_{\text{f}}$ and the 
propagation distance $d$ can become lower than the required limits. 
The deflections for stochastic interactions are parametrized by Van Ginneken 
in \cite{Van_Ginneken} with a direct calculation of the deflection in 
ionization using four-momentum conservation. 
Furthermore, there are parametrizations for deflections given in GEANT4 \cite{GEANT4_manual} for bremsstrahlung and photonuclear interaction, which 
are available in PROPOSAL, too.
To estimate the deflection along 
a continuous energy loss, multiple scattering established by Moliére 
\cite{moliere_scattering} and the gaussian approximation by Highland 
can be chosen \cite{highland_scattering}. 
The latest updates with a detailed description of the whole tool can be found 
in \cite{phd_soedingrekso}.
All simulations are done with PROPOSAL $7.3.0$.


The deflections per interaction are presented in Figure~\ref{fig:defl_per_int} 
for each interaction type and the total amount. A single deflection 
extend over several orders of magnitude with a median of $\SI{1.5e-6}{\degree}$
and a $\SI{95}{\percent}$ central interval of $[\SI{1.6e-7}{degree}, \,\SI{2.8e-4}{degree}]$.

\begin{figure}
    \centering 
    \includegraphics[width=0.8\textwidth]{figures/1PeV_1TeV_1000events.pdf}
    \caption{The propagation is done for $\num{1000}$ 
    muons from $E_{\text{i}} = \SI{1}{\peta\electronvolt}$ to $E_{\text{f,\,min}} = \SI{1}{\tera\electronvolt}$ using $\texttt{e\_cut} = \SI{500}{\mega\electronvolt}$ and $\texttt{v\_cut} = 0.05$. For multiple scattering 
    the Moliére parametrization is used. Details are presented in 
    Table~\ref{tab:defl_per_int}.}
    \label{fig:defl_per_int}
\end{figure}

\begin{table}
    \centering 
    \caption{The medians of deflections per interaction from Figure~\ref{fig:defl_per_int} are presented for each interaction type and the total distribution with the upper and lower limits of the $\SI{95}{\percent}$ 
    central content levels.}
    \begin{tabular}{ccccc}
        \toprule 
        brems & nuclint & epair & ioniz & total \\
        $\theta\,/\,\SI{e-5}{\degree}$ & $\theta\,/\,\SI{e-4}{\degree}$ & $\theta\,/\,\SI{e-6}{\degree}$ & $\theta\,/\,\SI{e-5}{\degree}$ & $\theta\,/\,\SI{e-6}{\degree}$\\
        \midrule 
        $3.8_{-0.1}^{+297}$ & $1.2_{-0.4}^{+96}$ & $1.3_{-0.2}^{+42}$ & $4.4_{-0.1}^{+181}$& $1.5_{-0.2}^{+279}$\\ 
        \bottomrule
    \end{tabular}
    \label{tab:defl_per_int}
\end{table}
\section{Accumulated Muon Deflection}\label{sec:accum_defl}

As shown in Section~\ref{sec:defl_per_int}, the deflection per interaction 
is lower $\SI{1}{\degree}$ and thus not relevant for the current muon 
directional reconstruction. Since these deflections accumulate along the 
propagation distance, the angle between the incoming muon and the outgoing 
muon direction is simulated. 
At first, the deflections in PROPOSAL are compared to 
the tools MUSIC and GEANT4.



\begin{figure}
    \centering
    \subcaptionbox{
        \label{fig:compare_MUSIC_degree}}
        {\includegraphics[width=0.48\textwidth]{figures/compare_MUSIC_degree.pdf}}
    \subcaptionbox{
        \label{fig:compare_MUSIC_dist}}
        {\includegraphics[width=0.48\textwidth]{figures/compare_MUSIC_dist.pdf}}
    \caption{}
    \label{fig:compare_MUSIC}
\end{figure}








For current analyses, it is important to study the impact of the muon 
deflection on the angular resolution to estimate a reconstruction uncertainty.
For this purpose, four different initial energies 
from $E_{\text{i}} = \SI{10}{\tera\electronvolt}$ to 
$E_{\text{i}} = \SI{10}{\peta\electronvolt}$ are used and the final 
energy is set to $E_{\text{f,\,min}} \geq \SI{10}{\giga\electronvolt}$ with 
$E_{\text{f,\,min}} < E_{\text{i}}$ for each simulation. To compare the results of 
a total of $\num{36}$ simulations, the median of the deflection distribution 
with a $\SI{95}{\percent}$ central interval is presented in 
Figure~\ref{fig:fit_median}.

\begin{equation}
    \label{eqn:fit_median}
\end{equation}


\begin{figure}
    \centering 
    \includegraphics[width=0.8\textwidth]{figures/fit_median_defl_cut_10percent_only_poly.pdf}
    \caption{The median of the accumulated deflection with a $\SI{95}{\percent}$ 
    central interval is shown for four different initial energies $E_{\text{i}}$. 
    Each data set includes more than $\num{50000}$ events with the requirement 
    that the true final particle energy $E_{\text{f}}$ is maximum 
    $\SI{10}{\percent}$ below the set final energy $E_{\text{f,\,min}}$,   
    $E_{\text{f}} > E_{\text{f,\,min}} \cdot 0.9$. The energy cuts are $\texttt{e\_cut} = \SI{500}{\mega\electronvolt}$ and $\texttt{v\_cut} = 0.05$. 
    Since the medians overlap for different initial energies, there is no 
    strong impact of the initial energy on the median deflection. These 
    medians can be fit by a third degree polynomial in the log-space as 
    shown in Equation~\ref{eqn:fit_median}. For energies 
    $E_{\text{f}} \approx \SI{500}{\giga\electronvolt} - \SI{1}{\tera\electronvolt}$, there is a minimal influence of deflection on the angular resolution of 
    KM3NeT \cite{KM3NeT_Resolution2016}. The resolution of IceCube is not 
    impacted \cite{IceCube_Resolution2021}.}
    \label{fig:fit_median}
\end{figure}
\section{Conclusion}\label{sec:conclusion}

At first, the recently into PROPOSAL implemented stochastic deflection is 
used to study the muon deflection per interaction. Mainly, the deflection 
is dominated by multiple scattering except for a few stochastic 
outliers by bremsstrahlung. These angles are lower $\sim\SI{1}{\degree}$. 

The results of PROPOSAL are tested against the common tools MUSIC and 
GEANT4 and all are in good agreement.
The median accumulated deflection depends on the final muon energy, primarily. 
This outcome is fit by a polynom and can simply be used for 
a theoretical estimation of the muon deflection in water and ice.
This deflection defines a lower limit on the directional resolution.
In the energy range of $\SI{500}{\giga\electronvolt} -- \SI{1}{\tera\electronvolt}$, there is a small impact of the muon deflection on the angular 
resolution of KM3NeT.

\textcolor{red}{relevance of muon-neutrino angle vs. muon deflection for higher energies}
% \begin{itemize}
%     \item deflection per interaction is negligible
%     \item accumulated deflection is in the order of magnitude of KM3NeT for low energy muons (GeV-TeV) 
%     \item median accumulated defl can be fit and used for simple deflection estimation, since deflection mainly depends on final particle energy
%     \item relevance of muon-neutrino angle vs. muon deflection for higher energies
% \end{itemize}
\input{06_acknowledgement.tex}

%% The Appendices part is started with the command \appendix;
%% appendix sections are then done as normal sections
%% \appendix

%% \section{}
%% \label{}

%% If you have bibdatabase file and want bibtex to generate the
%% bibitems, please use
%%
\alexander{Die Referenzen müssen überprüft werden; in vielen Fällen stehen
nur Initialen da, selbst für den Nachnamen!}
\bibliographystyle{elsarticle-num} 
\bibliography{lit.bib}


%% else use the following coding to input the bibitems directly in the
%% TeX file.

% \begin{thebibliography}{00}

%% \bibitem{label}
%% Text of bibliographic item

% \bibitem{bibtest}Peter Pan, Uni Dortmund
% \bibitem{b2}Pascal, 2. Test

% \end{thebibliography}
\end{document}
\endinput
%%
%% End of file `elsarticle-template-num.tex'.
