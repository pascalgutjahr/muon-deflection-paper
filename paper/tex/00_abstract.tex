For uncertainty estimation of reconstruction of the muon direction, the 
deflection of muons along a propagated path in ice and water is simulated using 
the tool PROPOSAL (Propagator with optimal precision and optimized speed for all leptons). 
The deflection per interaction spans several 
orders of magnitude with $\theta \in [\SI[print-unity-mantissa = false]{e-9}{\degree},\, \SI{1}{\degree}]$ 
for energies $E < \SI{1}{\peta\electronvolt}$ 
and is rather negligible. The accumulated 
deflection of a muon with an energy $E_{\mathrm{\mu}} = \SI{500}{\giga\electronvolt}$ 
after a propagated track is about $\theta_{\text{acc}} = 0.10_{-0.02}^{+0.27}\,\si{\degree}$ 
with a $\SI{95}{\percent}$ central interval, which is in the order of magnitude of 
the directional resolution of 
present neutrino detectors. This deflection can be fit against energy with a 
third degree polynomial. Furthermore, the deflections are compared to the results 
of the propagation tools MUSIC and GEANT4 and all deflections lead to 
a good agreement.

% Machine learning is also used to estimate the 
% deflection distribution for initial and final energy and propagation distance. 



