\section{Introduction}\label{sec:introduction}
Neutrinos can cross the universe all the way to earth, since they are 
nearly massless and not charged. On earth, neutrinos can interact 
with a nucleus and produce an electron, muon or tau in the charged current 
via the weak interaction \cite{}. These charged daughter particles are very 
high-energetic and emit Cherenkov light \cite{}, which can be detected 
by large neutrino telescopes like IceCube \cite{IceCube_Instrumentation} or 
KM3NeT \cite{KM3NeT_Design}. 
By the reconstruction of the direction of the daughter particles, the direction 
of the original neutrino can be inferred. The angular resolution is at its 
best for muons, since they can travel up to a few kilo meters due to their 
large mass, instead of the much lighter electrons and taus - which decay almost 
instantly. Current angular resolutions are 
$\Phi_{\text{I}} > \SI{0.1}{\degree} - \SI{0.3}{\degree}$ for energies 
$E \in [\SI{3}{\tera\electronvolt},\,\SI{3}{\peta\electronvolt}]$ in IceCube 
\cite{IceCube_Resolution2021} 
and 
$\Phi_{\text{K}} < \SI{0.2}{\degree}$ for $E > \SI{10}{\tera\electronvolt}$ in 
KM3NeT/ARCA \cite{KM3NeT_Resolution2021}.
Since muons do up to $\num{10000}$ interactions along their propagation while they 
are deflected in each interaction, it is important to study if the accumulated 
deflection along a track impacts the angular resolution of current 
neutrino detectors. 

For this purpose, the lepton propagation 
tool PROPOSAL \cite{koehne2013proposal, dunsch_2018_proposal_improvements} is used to study the muon deflection per interaction in 
Section~\ref{sec:defl_per_int} and the 
accumulated deflection in Section~\ref{sec:accum_defl}. The 
paper concludes with a summarized overview in Section~\ref{sec:conclusion}.


% Überlick: 
% Neutrinos kommen aus dem Universum -> fliegen auf Erde -> WW mit Kern 
% -> e,mu,tau entstehen -> geladene hochenergetische Teilchen -> Cherenkov Licht 
% -> kann mit Neutrinodetektoren (wie IceCube und KM3NeT) gemessen werden 
% -> Richtungsrekonstruktion ->
% Rückschluss auf Richtung, aus der das Neutrino entsendet wurde

% -> Myonen haben die beste Richtungsauflösung (Auflösung von IceCube und KM3NeT), 
% da sie aufgrund ihrer hohen Masse 
% eine deutlich größere Strecke zurücklegen können, als elektronen und taus, welche 
% sofort zerfallen -> da myonen entlang ihrer propagierten strecke bis zu 10.000 
% wechselwirkungen durchführen können und sie in jeder einzelnen abgelenkt wird, 
% gilt es nun zu überprüfen, ob die ablenkung der myonen die Richtungsauflösung 
% beeinflusst.

% To determine 
% the position of extraterrestrial neutrino sources more precisely, neutrino 
% experiments are constantly being optimized.

% Neutrinos have a major role to play in the study of the universe. 
% The direction of the original 
% neutrino - emitted by extraterrestrial sources - can be inferred by 
% reconstructing the direction of the daughter 
% particles, which are produced in charged currents.


