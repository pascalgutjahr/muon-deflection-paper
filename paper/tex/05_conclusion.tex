\section{Conclusion}\label{sec:conclusion}

At first, the recently into PROPOSAL implemented stochastic deflection is 
used to study the muon deflection per interaction. Mainly, the deflection 
is dominated by multiple scattering except for a few stochastic 
outliers by bremsstrahlung. These angles are lower $\sim\SI{1}{\degree}$. 

The results of PROPOSAL are tested against the common tools MUSIC and 
GEANT4 and all are in good agreement.
The median accumulated deflection depends on the final muon energy, primarily. 
This outcome is fit by a polynom and can simply be used for 
a theoretical estimation of the muon deflection in water and ice.
This deflection defines a lower limit on the directional resolution.
In the energy range of $\SI{500}{\giga\electronvolt} -- \SI{1}{\tera\electronvolt}$, there is a small impact of the muon deflection on the angular 
resolution of KM3NeT.

\textcolor{red}{relevance of muon-neutrino angle vs. muon deflection for higher energies}
% \begin{itemize}
%     \item deflection per interaction is negligible
%     \item accumulated deflection is in the order of magnitude of KM3NeT for low energy muons (GeV-TeV) 
%     \item median accumulated defl can be fit and used for simple deflection estimation, since deflection mainly depends on final particle energy
%     \item relevance of muon-neutrino angle vs. muon deflection for higher energies
% \end{itemize}